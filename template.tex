\documentclass{report}
\usepackage[utf8]{inputenc}

\usepackage{titling}
\usepackage{graphicx}
\usepackage[utf8]{inputenc}

\usepackage{style/a4-document-setup}
\usepackage{style/internship-styling}

\renewcommand{\versiondate}{16-02-2020}
\renewcommand{\versionnumber}{0.8}
\renewcommand{\versionname}{Testing version}

\author{Mies van der Lippe}
\title{LaTex Styling}

\begin{document}
	\begin{titlepage}

\thetitle
\theauthor

\end{titlepage}
	\tableofcontents
	\newpage
	\section{Introduction}
	This is an introduction. 
	
	\newpage
	
	\section{The First Section}
	This is text within a section. 
	
	\subsection{The first subsection}
	This is a subsection.
	
	\subsubsection{A sub-sub section}
	This shouldn't be numbered. 
	
	\section{Tabular}
	Tables should have stylized headers. 
	
	\begin{table}[!h]
		\begin{tabular}{|p{2cm}|p{2.5cm}|p{11cm}|} \hline
			Column 1	& Column2	& Some longer content\\\hline
			Row1	& Date1	& Table with some content\\\hline	
			Row2	& Date2	& Table with some content that overflows onto the next line. It should be fairly long to make it to the next line.\\\hline	
			Row3	& Date3	& Table with some content\\\hline	
		\end{tabular}
		\caption{This is a table caption}
	\end{table} 

	\section{Paragraphs}
	This is the first paragraph, contains some text to test the paragraph
	interlining, paragraph indentation and some other features. Also, is 
	easy to see how new paragraphs are defined by simply entering a double 
	blank space.
	
	Hello,  here  is  some  text  without  a  meaning.   This  text  should
	show what a printed text will look like at this...
	
	\section{Custom functions}
	This styling / template provides some custom functions. 
	
	\subsection{Custom labels}
	This function allows user control of label contents. 
	
	\customlabel{labeltarget}{This is a label target that's not the section title!}
	
	Click here to go to the label;
	
	\nameref{labeltarget}
	
	\newpage
		
	\section{Custom numbered lists}
	In order to label products I've written a few simple commands that allow users to pre-fix and automatically 
	label lists. 
	
	\begin{itemize}
		\prefixeditem{a-descriptibe-label}{A}{This is the first item in a list}\\
		This is a pre-fixed item. See how it's automatically labeled A1? No need to manually label your products
		anymore. 
				
		\prefixedsubitems{
			\prefixedsubitem{nested-list-item}{A}{Nested list!}\\
			There's a function for nested lists too!
			
			\prefixedsubitem{beauty}{A}{Beauty}\\
			They work pretty well. 
		}
	
		\prefixeditem{roll-over-demo}{A}{Automatic roll-over for subitems}\\
		See how sub-items start over at 1?
		\prefixedsubitems{
			\prefixedsubitem{rolled-over-item}{A}{Testing 1 2}\\
			No need to do that manually. 
		}
		
	\end{itemize}
	
	There's no automatic rollover on primary items. 
	
	\begin{itemize}
		\prefixeditem{continuation}{A}{We're continueing to count}\\
		This should be item 3. 
	\end{itemize}

	You can manually reset the counters with a simple command. 
	
	\resetprefixeditemcounters{}
	\begin{itemize}
		\prefixeditem{continuation-b}{B}{New list, new numbers}\\
		This should be item 1. 
	\end{itemize}
	
	You can have some more control over the numbering by setting the counters manually. They're increased on every
	item or subitem. The subitem counter is reset on incrementing the primary counter. 
	
	This is a reference to our product "\nameref{continuation-b}". You can now see what it's value is. When we re-order
	the list, or add products all references and counts are updated automatically. 
	
	\listoffigures
	\listoftables
	
	
\end{document}